%% ## Construye tu propia portada ##
%% 
%% Una portada se conforma por una secuencia de "Blocks" que incluyen
%% piezas individuales de informaci'on. Un "Block" puede incluir, por
%% ejemplo, el t'itulo del documento, una im'agen (logotipo de la universidad),
%% el nombre del autor, nombre del supervisor, u cualquier otra pieza de
%% informaci'on.
%%
%% Cada "Block" aparece centrado horizontalmente en la p'agina y,
%% verticalmente, todos los "Blocks" se distruyen de manera uniforme 
%% a lo largo de p'agina.
%%
%% Nota tambi'en que, dentro de un mismo "Block" se pueden cortar
%% lineas usando el comando \\
%%
%% El tama'no del texto dentro de un "Block" se puede modificar usando uno de
%% los comandos:
%%   \small      \LARGE
%%   \large      \huge
%%   \Large      \Huge
%%
%% Y el tipo de letra se puede modificar usando:
%%   \bfseries - negritas
%%   \itshape  - it'alicas
%%   \scshape  - small caps
%%   \slshape  - slanted
%%   \sffamily - sans serif
%%
%% Para producir plantillas generales, la informaci'on que ha sido inclu'ida
%% en el archivo principal "tesis.tex" se puede accesar aqu'i usando:
%%   \insertauthor
%%   \inserttitle
%%   \insertsupervisor
%%   \insertinstitution
%%   \insertdegree
%%   \insertfaculty
%%   \insertdepartment
%%   \insertsubmitdate

\begin{titlepage}
  \TitleBlock{\bfseries\LARGE\insertinstitution}
  \TitleBlock[\bigskip]{\scshape\insertcampus}
  \TitleBlock{\bigskip\includegraphics[scale=0.5]{\insertinstitutionImage}}
  \TitleBlock{\Huge\scshape\inserttitle}
  \TitleBlock{\bigskip\scshape
    Que para obtener el t\'itulo de \\
    \insertdegree \\
    Presenta}
  \TitleBlock{\scshape\LARGE\insertauthor}
  \TitleBlock{\raggedright\bigskip\scshape
    Director de Tesis\\
    \insertsupervisor}
  \TitleBlock{Texcoco, Estado de M\'exico, Mayo \insertsubmitdate}
\end{titlepage}
