\documentclass[oneside,numbers,spanish]{./ezthesis}
%% # Opciones disponibles para el documento #
%%
%% Las opciones con un (*) son las opciones predeterminadas.
%%
%% Modo de compilar:
%%   draft            - borrador con marcas de fecha y sin im'agenes
%%   draftmarks       - borrador con marcas de fecha y con im'agenes
%%   final (*)        - version final de la tesis
%%
%% Tama'no de papel:
%%   letterpaper (*)  - tama'no carta (Am'erica)
%%   a4paper          - tama'no A4    (Europa)
%%
%% Formato de impresi'on:
%%   oneside          - hojas impresas por un solo lado
%%   twoside (*)      - hijas impresas por ambos lados
%%
%% Tama'no de letra:
%%   10pt, 11pt, o 12pt (*)
%%
%% Espaciado entre renglones:
%%   singlespace      - espacio sencillo
%%   onehalfspace (*) - espacio de 1.5
%%   doublespace      - a doble espacio
%%
%% Formato de las referencias bibliogr'aficas:
%%   numbers          - numeradas, p.e. [1]
%%   authoryear (*)   - por autor y a'no, p.e. (Newton, 1997)
%%
%% Opciones adicionales:
%%   spanish         - tesis escrita en espa'nol
%%
%% Desactivar opciones especiales:
%%   nobibtoc   - no incluir la bibiolgraf'ia en el 'Indice general
%%   nofancyhdr - no incluir "fancyhdr" para producir los encabezados
%%   nocolors   - no incluir "xcolor" para producir ligas con colores
%%   nographicx - no incluir "graphicx" para insertar gr'aficos
%%   nonatbib   - no incluir "natbib" para administrar la bibliograf'ia

%% Paquetes adicionales requeridos se pueden agregar tambi'en aqu'i.
%% Por ejemplo:
%\usepackage{subfig}
%\usepackage{multirow}

%% # Datos del documento #
%% Nota que los acentos se deben escribir: \'a, \'e, \'i, etc.
%% La letra n con tilde es: \~n.


% University Data
\institution{Universidad Aut\'onoma del Estado de M\'exico}
\campus{Centro Universitario UAEM Texcoco}
\institutionImage{assets/images/general/UAEM}
%\faculty{}
\department{Departamento de Sistemas Computacionales}

% Student Data
\author{Ricardo Alberto P\'erez Candelas}
\title{Implementaci\'on de un caso de estudio con Testing Driven Development}

% General Data
\documentType{Protocolo de Tesis}
\degree{Ingeniero en Computaci\'on}
\supervisor{Irene Aguilar Juar\'ez}

%\geometry{top=40mm,bottom=33mm,inner=40mm,outer=25mm}

\hyperlinking
\begin{document}

%% ## Construye tu propia portada ##
%% 
%% Una portada se conforma por una secuencia de "Blocks" que incluyen
%% piezas individuales de informaci'on. Un "Block" puede incluir, por
%% ejemplo, el t'itulo del documento, una im'agen (logotipo de la universidad),
%% el nombre del autor, nombre del supervisor, u cualquier otra pieza de
%% informaci'on.
%%
%% Cada "Block" aparece centrado horizontalmente en la p'agina y,
%% verticalmente, todos los "Blocks" se distruyen de manera uniforme 
%% a lo largo de p'agina.
%%
%% Nota tambi'en que, dentro de un mismo "Block" se pueden cortar
%% lineas usando el comando \\
%%
%% El tama'no del texto dentro de un "Block" se puede modificar usando uno de
%% los comandos:
%%   \small      \LARGE
%%   \large      \huge
%%   \Large      \Huge
%%
%% Y el tipo de letra se puede modificar usando:
%%   \bfseries - negritas
%%   \itshape  - it'alicas
%%   \scshape  - small caps
%%   \slshape  - slanted
%%   \sffamily - sans serif
%%
%% Para producir plantillas generales, la informaci'on que ha sido inclu'ida
%% en el archivo principal "tesis.tex" se puede accesar aqu'i usando:
%%   \insertauthor
%%   \inserttitle
%%   \insertsupervisor
%%   \insertinstitution
%%   \insertdegree
%%   \insertfaculty
%%   \insertdepartment
%%   \insertsubmitdate

\begin{titlepage}
  \TitleBlock{\bfseries\LARGE\insertinstitution}
  \TitleBlock[\bigskip]{\scshape\insertcampus}
  \TitleBlock{\bigskip\includegraphics[scale=0.5]{\insertinstitutionImage}}
  \TitleBlock{\Huge\scshape\inserttitle}
  \TitleBlock{\bigskip\scshape
    Que para obtener el t\'itulo de \\
    \insertdegree \\
    Presenta}
  \TitleBlock{\scshape\LARGE\insertauthor}
  \TitleBlock{\raggedright\bigskip\scshape
    Director de Tesis\\
    \insertsupervisor}
  \TitleBlock{Texcoco, Estado de M\'exico, Mayo \insertsubmitdate}
\end{titlepage}



\tableofcontents
%\listoffigures
%\listoftables

% Chapters

\chapter{Introducci'on}


\chapter{Problem'atica}

\chapter{Justificaci'on}

\chapter{Objetivos}

\section{Generales}
\setlength{\parindent}{10ex}
Diseñar e Implementar un Sistema Web que administre una comunidad de profesionistas que deseen trabajar en colaboración en proyectos semi-profesionales, mediante 
el paradigma  Testing Driven Development

\section{Particulares}
\begin{enumerate}
\item Identificar los roles y los procesos involucrados en las comunidades de trabajo colaborativo.
\item Modelar los procesos y la base de información  necesarios para dar soporte al sistema
\item Documentar y aplicar los principios del paradigma Testing Driven Develpment
\item Seleccionar el software que soporta el paradigma
\end{enumerate}

\chapter{Hip'otesis o supuesto}

\chapter{Metodolog'ia}

\chapter{Cronograma}

\chapter{'Indice tentativo}

\chapter{Bibliograf\'ia}
\nocite{*}




\appendix

\bibliography{./biblio}

\end{document}
